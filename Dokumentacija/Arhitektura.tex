\chapter{Arhitektura i dizajn sustava}
		
		%\textbf{\textit{dio 1. revizije}}\\

		%\textit{ Potrebno je opisati stil arhitekture te identificirati: podsustave, preslikavanje na radnu platformu, spremišta podataka, mrežne protokole, globalni upravljački tok i sklopovsko-programske zahtjeve. Po točkama razraditi i popratiti odgovarajućim skicama:}
	%\begin{itemize}
		%\item 	\textit{izbor arhitekture temeljem principa oblikovanja pokazanih na predavanjima (objasniti zašto ste baš odabrali takvu arhitekturu)}
		%\item 	\textit{organizaciju sustava s najviše razine apstrakcije (npr. klijent-poslužitelj, baza podataka, datotečni sustav, grafičko sučelje)}
		%\item 	\textit{organizaciju aplikacije (npr. slojevi frontend i backend, MVC arhitektura) }		
	%\end{itemize}

	\noindent{Arhitektura se dijeli na tri podsustava:}
	\begin{packed_item}
		\item web poslužitelj
		\item web aplikaciju
		\item bazu podataka
	\end{packed_item}
	\noindent{Web aplikacije često zahtijevaju integraciju poslužitelja, aplikacije i baze podataka kako bi pružile korisnicima dinamičke i interaktivne sadržaje. \textbf{Web poslužitelj} služi kao središnja točka koja prima zahtjeve korisnika putem internetskog preglednika. \textbf{Web aplikacija} obrađuje zahtjev te ovisno o njemu komunicira s bazom podataka kako bi dohvatila, mijenjala ili pohranjivala podatke koji se koriste u procesu. \textbf{Baza podataka} je skup strukturiranih podataka koji se čuvaju na poslužitelju, a aplikacija je odgovorna za upravljanje tim podacima i osiguravanje njihove konzistentnosti. Kada se podaci mijenjaju putem aplikacije, te promjene se ažuriraju u bazi podataka, a zatim web poslužitelj šalje ažurirane informacije korisnicima putem preglednika.}
	\newline
	\noindent{Jezici korišteni pri izradi aplikacije su C\#, CSS, HTML i JavaScript.}
	\newline
	\noindent{Arhitektura sustava se bazira na \textbf{MVC} konceptu. MVC (Model-View-Controller) je arhitekturni obrazac za razvoj softverskih aplikacija. \textbf{Model} predstavlja podatke i logiku aplikacije, \textbf{View} predstavlja sučelje preko kojeg korisnici komuniciraju s aplikacijom i prikazuje im podatke iz Modela, dok \textbf{Controller} upravlja komunikacijom između Modela i View-a. Ovaj koncept omogućava jasnu organizaciju koda, razdvajanje odgovornosti između komponenti aplikacije (Model, View, Controller) te olakšava timsku suradnju i održavanje aplikacije. Zbog jasne podjele na modele, poglede i kontrolere, aplikacije izgrađene na MVC arhitekturi često su fleksibilnije i lakše za održavanje.}

	\pagebreak
				
		\section{Baza podataka}
			
			%\textbf{\textit{dio 1. revizije}}\\
			
		%\textit{Potrebno je opisati koju vrstu i implementaciju baze podataka ste odabrali, glavne komponente od kojih se sastoji i slično.}
		Za pohranu podataka iz naše web aplikacije napravljena je relacijska baza podataka. Tablicama i njihovim atributima ostvarena je komunikacija između organiziranosti, preciznosti i jednostavnog dohvaćanja podataka za daljnu obradu. Baza podataka je kreirana u SQLite-u, upravo zbog toga što se podaci iz baze mogu dijeliti između više računala i jer koristi jednu datoteku za pohranu cijele baze podataka.\\
		Baza podataka se sastoji od sljedećih entiteta:
		\begin{packed_item}
			\item User
			\item Regular
			\item Shelter
			\item TypeOfUser
			\item Communication
			\item Ad
			\item PhotoAd
			\item Pet
			\item ColorPet
			\item hasColor
		\end{packed_item}
		
			\subsection{Opis tablica}
			

				%\textit{Svaku tablicu je potrebno opisati po zadanom predlošku. Lijevo se nalazi točno ime varijable u bazi podataka, u sredini se nalazi tip podataka, a desno se nalazi opis varijable. Svjetlozelenom bojom označite primarni ključ. Svjetlo plavom označite strani ključ}
				\textbf{Korisnik (User)}
				Ovaj entitet sadržava sve primarne informacije o korisniku. Sadrži atribute: userID, userName, email, phoneNum i psw. Ovaj entitet u vezi je \textit{One-to-Many} s entitetom Ad (oglas), \textit{One-to-Many} s entitetom Communication (Komunikacija) oba preko atributa userID korisnika te u vezi \textit{One-to-One} s entitetom Regular (redovni korisnik), \textit{One-to-One} s Shelter(sklonište) i \textit{One-to-One} s TypeOfUser (tip korisnika) preko userID - korisničkog imena.
				
				
				\begin{longtblr}[
					label=none,
					entry=none
					]{
						width = \textwidth,
						colspec={|X[6,l]|X[6, l]|X[20, l]|}, 
						rowhead = 1,
					} %definicija širine tablice, širine stupaca, poravnanje i broja redaka naslova tablice
					\hline \SetCell[c=3]{c}{\textbf{User}}	 \\ \hline[3pt]
					\SetCell{LightGreen} userID & INTEGER	& jedinstveni identifikator korisnika	\\ \hline
					userName & VARCHAR & naziv korisnika u aplikaciji \\ \hline 
					email & VARCHAR & e-mail adresa korisnika\\ \hline 
					phoneNum & VARCHAR	&  broj telefona korisnika\\ \hline
					psw & VARCHAR & lozinka korisnika\\ \hline
					%\SetCell{LightBlue} primjer	& VARCHAR &   	\\ \hline 
				\end{longtblr}
				
				\textbf{Redovni korisnik (Regular)}
				Ovaj entitet sadržava informacije o regularnom korisniku. Sadrži atribute: userID, firstName - ime i lastName - prezime korisnika. Ovaj entitet u vezi je \textit{One-to-One} s entitetom User (Korisnik) preko userID korisnika.
				
				
				\begin{longtblr}[
					label=none,
					entry=none
					]{
						width = \textwidth,
						colspec={|X[6,l]|X[6, l]|X[20, l]|}, 
						rowhead = 1,
					} %definicija širine tablice, širine stupaca, poravnanje i broja redaka naslova tablice
					\hline \SetCell[c=3]{c}{\textbf{Regular}}	 \\ \hline[3pt]
					\SetCell{LightBlue} userID & INTEGER	& jedinstveni identifikator korisnika, (user.userID)	\\ \hline
					firstName & VARCHAR & ime korisnika \\ \hline 
					lastName & VARCHAR & prezime korisnika\\ \hline 
					%\SetCell{LightBlue} primjer	& VARCHAR &   	\\ \hline 
				\end{longtblr}
				
				\textbf{Sklonište (Shelter)}
				Ovaj entitet sadržava informacije o skloništu kao korisniku. Sadrži atribute: userID i nameShelter - naziv skloništa. Ovaj entitet u vezi je \textit{One-to-One} s entitetom User (Korisnik) preko userID korisnika.
				
				
				\begin{longtblr}[
					label=none,
					entry=none
					]{
						width = \textwidth,
						colspec={|X[6,l]|X[6, l]|X[20, l]|}, 
						rowhead = 1,
					} %definicija širine tablice, širine stupaca, poravnanje i broja redaka naslova tablice
					\hline \SetCell[c=3]{c}{\textbf{Shelter}}	 \\ \hline[3pt]
					\SetCell{LightBlue} userID & INTEGER	& jedinstveni identifikator korisnika, (user.userID)	\\ \hline
					nameShelter & VARCHAR & naziv skloništa za životinje \\ \hline 
					
				\end{longtblr}
				
				\textbf{Tip korisnika (TypeOfUser)}
				Ovaj entitet sadržava informacije o tipu korisnika. Korisnik može biti regular - uobičajen korisnik ili shelter - sklonište za životinje. Sadrži atribute: userID i userType - tip korisnika. Ovaj entitet u vezi je \textit{One-to-One} s entitetom User (Korisnik) preko userID korisnika.
				
				
				\begin{longtblr}[
					label=none,
					entry=none
					]{
						width = \textwidth,
						colspec={|X[6,l]|X[6, l]|X[20, l]|}, 
						rowhead = 1,
					} %definicija širine tablice, širine stupaca, poravnanje i broja redaka naslova tablice
					\hline \SetCell[c=3]{c}{\textbf{TypeOfUser}}	 \\ \hline[3pt]
					\SetCell{LightBlue} userID & INTEGER	& jedinstveni identifikator korisnika, (user.userID)	\\ \hline
					userType & VARCHAR & tip korisnika - može biti shelter ili regular \\ \hline 
					 
				\end{longtblr}
				
				\textbf{Komunikacija (Communication)}
				Ovaj entitet sadržava sve važne informacije o komunikaciji ispod aktivnog oglasa. Sadrži atribute: textID, textCom - poruka, locCom - lokacija poruke te photoCom - slika unutar poruke te adID i userID - strani ključevi tablice Ad i User. Ovaj entitet u vezi je \textit{Many-to-One} s entitetom Ad (oglas) preko atributa adID oglasa i \textit{Many-to-One} s User preko userID korisnika.
				
				
				\begin{longtblr}[
					label=none,
					entry=none
					]{
						width = \textwidth,
						colspec={|X[6,l]|X[6, l]|X[20, l]|}, 
						rowhead = 1,
					} %definicija širine tablice, širine stupaca, poravnanje i broja redaka naslova tablice
					\hline \SetCell[c=3]{c}{\textbf{Communication}}	 \\ \hline[3pt]
					\SetCell{LightGreen} textID & INTEGER & jedinstveni identifikator poruke	\\ \hline
					photoCom & VARCHAR & slika poruke\\ \hline 
					textCom & VARCHAR & tekst poruke\\ \hline 
					locCom & VARCHAR	&  lokacija poruke\\ \hline
					\SetCell{LightBlue} adID	& INTEGER &  jedinstveni identifikator oglasa, (ad.adID)	\\ \hline 
					\SetCell{LightBlue} userID	& INTEGER & jedinstveni identifikator korisnika, (user.userID) 	\\ \hline
				\end{longtblr}
				
				\textbf{Oglas (Ad)}
				Ovaj entitet sadržava sve važne informacije o postavljanju oglasa. Sadrži atribute: adID, catID - kategorija oglasa, userID - strani ključ tablice User. Ovaj entitet u vezi je \textit{Many-to-One} s entitetom User (korisnik) preko atributa userID korisnika, \textit{One-to-Many} s Communication (Komunikacija) preko atributa adID, \textit{One-to-Many} s PhotoAd (slike oglasa) preko adID,
				\textit{One-to-One} s entitetom Pet(Ljubimac) preko adID oglasa.
				
				
				\begin{longtblr}[
					label=none,
					entry=none
					]{
						width = \textwidth,
						colspec={|X[6,l]|X[6, l]|X[20, l]|}, 
						rowhead = 1,
					} %definicija širine tablice, širine stupaca, poravnanje i broja redaka naslova tablice
					\hline \SetCell[c=3]{c}{\textbf{Ad}}	 \\ \hline[3pt]
					\SetCell{LightGreen} adID & INTEGER	& jedinstveni identifikator oglasa	\\ \hline
					catAd & VARCHAR & kategorije oglasa \\ \hline 
					\SetCell{LightBlue} userID	& INTEGER & jedinstveni identifikator korisnika, (user.userID)    	\\ \hline 
				\end{longtblr}
				
				\textbf{Slike oglas (PhotoAd)}
				Ovaj entitet sadržava sve važne informacije o postavljanju slika ispod oglasa. Sadrži atribute: photoID, photo - fotografija izgubljenog ljubimca, userID - strani ključ tablice User. Ovaj entitet u vezi je \textit{Many-to-One} s entitetom Ad (Oglas) preko atributa adID. Uz to, dodano je i ograničenje na razini baze - trigger za postavljanje maksimalno 3 slika uz oglas.
				
				
				\begin{longtblr}[
					label=none,
					entry=none
					]{
						width = \textwidth,
						colspec={|X[6,l]|X[6, l]|X[20, l]|}, 
						rowhead = 1,
					} %definicija širine tablice, širine stupaca, poravnanje i broja redaka naslova tablice
					\hline \SetCell[c=3]{c}{\textbf{PhotoAd}}	 \\ \hline[3pt]
					\SetCell{LightGreen} photoID & INTEGER	& jedinstveni identifikator slike oglasa	\\ \hline
					photo & VARCHAR & fotografija ljubimca \\ \hline 
					\SetCell{LightBlue} adID	& INTEGER &  jedinstveni identifikator oglasa, (ad.adID) 	\\ \hline 
				\end{longtblr}
				
				\textbf{Ljubimac (Pet)}
				Ovaj entitet sadržava sve važne informacije o nestalom kućnom ljubimcu. Sadrži atribute: petID, namePet - ime ljubimca, dateHourMis - datum i sat nestanka, location - lokacija pri postavljanju oglasa, vrsta ljubimca species, starost age, tekstni opis description i strani ključ adID iz Ad. Ovaj entitet u vezi je \textit{One-to-One} s entitetom Ad (oglas) preko atributa adID, \textit{Many-to-Many} s ColorPet (Komunikacija) preko tablice Has.
				
				
				\begin{longtblr}[
					label=none,
					entry=none
					]{
						width = \textwidth,
						colspec={|X[6,l]|X[6, l]|X[20, l]|}, 
						rowhead = 1,
					} %definicija širine tablice, širine stupaca, poravnanje i broja redaka naslova tablice
					\hline \SetCell[c=3]{c}{\textbf{Pet}}	 \\ \hline[3pt]
					\SetCell{LightGreen} petID & INTEGER & jedinstveni identifikator ljubimca	\\ \hline
					namePet & VARCHAR & ime na koje se odaziva ljubimac \\ \hline 
					dateHourMis & DATETIME & datum i sat nestanka ljubimca\\ \hline 
					location & VARCHAR	&  geolokacija izgubljenog ljubimca\\ \hline
					species & VARCHAR & vrsta ljubimca\\ \hline
					age & VARCHAR & starost ljubimca\\ \hline
					description & VARCHAR & opis ljubimca\\ \hline
					\SetCell{LightBlue} adID	& INTEGER &  jedinstveni identifikator oglasa, (ad.adID) 	\\ \hline 
				\end{longtblr}
				
				\textbf{Boja ljubimca (ColorPet)}
				Ovaj entitet sadržava sve važne informacije o boji kućnog ljubimca. Sadrži atribute: colorID i boja ljubimca color. Ovaj entitet u vezi je \textit{Many-to-Many} s entitetom Pet(Ljubimac) preko tablice Has.
				
				
				\begin{longtblr}[
					label=none,
					entry=none
					]{
						width = \textwidth,
						colspec={|X[6,l]|X[6, l]|X[20, l]|}, 
						rowhead = 1,
					} %definicija širine tablice, širine stupaca, poravnanje i broja redaka naslova tablice
					\hline \SetCell[c=3]{c}{\textbf{ColorPet}}	 \\ \hline[3pt]
					\SetCell{LightGreen} colorID & INTEGER	& jedinstveni identifikator boje ljubimca	\\ \hline
					color & VARCHAR & boja ljubimca \\ \hline 
					
				\end{longtblr}
				
				\textbf{ImaBoju (Has)}
				Ovaj entitet spaja boju i kućnog ljubimca. Sadrži atribute jedinstvene šifre colorID i petID. Ovaj entitet u vezi je \textit{Many-to-Many} s entitetom Pet(Ljubimac) preko atributa petID te u vezi \textit{Many-to-Many} s entitetom ColorPet(boja ljubimca) preko atributa colorID.
				
				
				
				\begin{longtblr}[
					label=none,
					entry=none
					]{
						width = \textwidth,
						colspec={|X[6,l]|X[6, l]|X[20, l]|}, 
						rowhead = 1,
					} %definicija širine tablice, širine stupaca, poravnanje i broja redaka naslova tablice
					\hline \SetCell[c=3]{c}{\textbf{Has}}	 \\ \hline[3pt]
					\SetCell{LightBlue} petID	& INTEGER &  jedinstveni identifikator ljubimca, (pet.petID) \\ \hline
					\SetCell{LightBlue} colorID	& INTEGER &  jedinstveni identifikator boje ljubimca, (colorPet.colorID) 	\\ \hline 
					 
				\end{longtblr}
				
				
			
				
			
			\subsection{Dijagram baze podataka}
				%\textit{ U ovom potpoglavlju potrebno je umetnuti dijagram baze podataka. Primarni i strani ključevi moraju biti označeni, a tablice povezane. Bazu podataka je potrebno normalizirati. Podsjetite se kolegija "Baze podataka".}
				\begin{figure}
					\centering
					\includegraphics[width=0.8\linewidth]{ERbaza.png}
					\caption{ER dijagram baze podataka}
					\label{fig:your_label}
				\end{figure}
			
			\eject
			
			
		\section{Dijagram razreda}
		
			%\textit{Potrebno je priložiti dijagram razreda s pripadajućim opisom. Zbog preglednosti je moguće dijagram razlomiti na više njih, ali moraju biti grupirani prema sličnim razinama apstrakcije i srodnim funkcionalnostima.}\\
			
			%\textbf{\textit{dio 1. revizije}}\\
			
			%\textit{Prilikom prve predaje projekta, potrebno je priložiti potpuno razrađen dijagram razreda vezan uz \textbf{generičku funkcionalnost} sustava. Ostale funkcionalnosti trebaju biti idejno razrađene u dijagramu sa sljedećim komponentama: nazivi razreda, nazivi metoda i vrste pristupa metodama (npr. javni, zaštićeni), nazivi atributa razreda, veze i odnosi između razreda.}\\
			
			%\textbf{\textit{dio 2. revizije}}\\			
			
			%\textit{Prilikom druge predaje projekta dijagram razreda i opisi moraju odgovarati stvarnom stanju implementacije}
			
			Razredi modela odražavaju strukturu podataka unutar aplikacije, a metode koje su implementirane unutar njih služe za izravnu komunikaciju s bazom podataka kako bi dobili ili manipulirali traženim podacima. Razred Klijent predstavlja registriranog korisnika sustava s mogućnošću pristupa osnovnim funkcionalnostima sustava. Razred Sklonište ima sve funkcionalnosti kao Klijent, uz dodatnu mogućnost dodavanja kategorija oglasa. Razredi Ljubimac i Oglas sadrže informacije potrebne za stvaranje oglasa koji će biti prikazani svim korisnicima.
			
			\begin{figure}[H]
				
				\includegraphics[scale =0.4]{kontroler.JPEG}
				\centering
				\caption{Dijagram razreda - dio Controllers}
				\label{fig:your_label}
			\end{figure}
			
			\begin{figure}[H]
				
				\includegraphics[scale =0.4]{model.JPEG}
				\centering
				\caption{Dijagram razreda - dio Models}
				\label{fig:your_label}
			\end{figure}
			%\eject
		
		\section{Dijagram stanja}
			
			
			\textbf{\textit{dio 2. revizije}}\\
			
			\textit{Potrebno je priložiti dijagram stanja i opisati ga. Dovoljan je jedan dijagram stanja koji prikazuje \textbf{značajan dio funkcionalnosti} sustava. Na primjer, stanja korisničkog sučelja i tijek korištenja neke ključne funkcionalnosti jesu značajan dio sustava, a registracija i prijava nisu. }
			
			
			\eject 
		
		\section{Dijagram aktivnosti}
			
			\textbf{\textit{dio 2. revizije}}\\
			
			 \textit{Potrebno je priložiti dijagram aktivnosti s pripadajućim opisom. Dijagram aktivnosti treba prikazivati značajan dio sustava.}
			
			\eject
		\section{Dijagram komponenti}
		
			\textbf{\textit{dio 2. revizije}}\\
		
			 \textit{Potrebno je priložiti dijagram komponenti s pripadajućim opisom. Dijagram komponenti treba prikazivati strukturu cijele aplikacije.}