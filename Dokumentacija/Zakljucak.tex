\chapter{Zaključak i budući rad}
		
		%\textbf{\textit{dio 2. revizije}}\\
		
		 %\textit{U ovom poglavlju potrebno je napisati osvrt na vrijeme izrade projektnog zadatka, koji su tehnički izazovi prepoznati, jesu li riješeni ili kako bi mogli biti riješeni, koja su znanja stečena pri izradi projekta, koja bi znanja bila posebno potrebna za brže i kvalitetnije ostvarenje projekta i koje bi bile perspektive za nastavak rada u projektnoj grupi.}
		
		 %\textit{Potrebno je točno popisati funkcionalnosti koje nisu implementirane u ostvarenoj aplikaciji.}
		
		Zadatak grupe bio je razvoj aplikacije za pomoć u potrazi izgubljenih kućnih ljubimaca uz mogućnost postavljanja oglasa o nestaloj životinji, pregled oglasa i komentiranje oglasa. Aplikacija je napravljena  u 14 tjedana, koji su podijeljeni u dvije faze.\\
		\indent U prvoj fazi okupljen je tim za razvoj aplikacije od 7 članova i dodijeljen projektni zadatak. U ovoj fazi najintenzivniji rad je proveden na dokumentiranju zahtjeva, koje je bilo osnova za što lakšu implementaciju osmišljenog sustava. Specifikacija programske potpore, koja uključuje funkcionalne zahtjeve, obrasce uporabe i različite vrste UML dijagrama (dijagram obrazaca uporabe, sekvencijski dijagrami, dijagrami razreda, model baze podataka) služili su kao nacrti i orijentir za izradu frontenda i backenda.\\
		\indent Druga faza projekta bila je intenzivnija po pitanju implementacije. Kod implementacije programskog rješenja nije bilo većih problema zbog spremnosti članova grupe na pomoć i samostalno učenje. U ovoj fazi izrađeni su preostali dijagrami (dijagram stanja, aktivnosti, komponenti i razmještaja) te ostatak dokumentacije.\\
		\indent Sudjelovanje na ovom projektu bilo je vrijedno iskustvo svim članovima grupe jer nas se većina po prvi put susrela s ovakvom vrstom zadatka u kojem je bila potrebna visoka razina organizacije, spremnost za rad u timu, samostalno učenje, disciplina i redovit rad. Zaključno, članovi tima su zadovoljni izrađenom aplikacijom te su implementirane sve planirane funkcionalnosti.
		
		
		\eject 