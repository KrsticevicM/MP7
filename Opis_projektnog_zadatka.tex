\chapter{Opis projektnog zadatka}
		
		%\textbf{\textit{dio 1. revizije}}\\
		
		%\textit{Na osnovi projektnog zadatka detaljno opisati korisničke zahtjeve. Što jasnije opisati cilj projektnog zadatka, razraditi problematiku zadatka, dodati nove aspekte problema i potencijalnih rješenja. Očekuje se minimalno 3, a poželjno 4-5 stranica opisa.	Teme koje treba dodatno razraditi u ovom poglavlju su:}\\
		%\underline{CILJ}\\
		Cilj ovog projektnog zadatka je razviti web aplikaciju „Nestali ljubimci“ koja će korisniku olakšati potragu za odlutalim kućnim ljubimcem. Budući da je informacije o potrazi nepraktično oglašavati putem uobičajenih internetskih platformi za komunikaciju (npr. društvene mreže, stranice za oglašavanje, forumi i slično), ova web aplikacija će vlasnicima životinja, skloništima za životinje i dobrim ljudima koji pomažu u pronalasku životinje omogućiti brz i efikasan način izmjene informacija o nestanku, opažanju i pronalasku odbjeglog kućnog ljubimca. U ime što bržeg pronalaska ljubimca, web aplikacija će biti razvijena za mobilne uređaje  kako bi u procesu potrage podržala veću brzinu reagiranja, koja je često ključna za uspješan završetak potrage za kućnim ljubimcem.\\
		
		Prilikom ulaska u web aplikaciju prikazuju se aktivni oglasi i oglasi o životinjama koje su pronašla skloništa.\\
		
		\textbf{KORISNICI}\\
		Aplikacija podržava rad tri tipa korisnika (neregistrirani korisnik, registrirani korisnik i sklonište za životinje).\\
		\underline{Neregistrirani korisnik} pri ulasku u web aplikaciju ima mogućnost pregledavati i pretraživati aktivne oglase o nestalim kućnim ljubimcima i oglase skloništa za životinje. Pretraživanje je ostvareno po nazivu skloništa i po kategorijama podataka o ljubimcu koje su dostupne pri oglašavanju (vrsta, ime na koje se odaziva, datum i sat nestanka, lokacija nestanka, boja, starost, tekstni opis). Za detaljniji pregled informacija o kućnom ljubimcu kao i pregled komunikacije o potrazi za njime potrebno je odabrati oglas. Ako neregistrirani korisnik može i želi doprinijeti potrazi davanjem nekih informacija, potrebna je registracija.
		Podaci potrebni za registraciju su:
		\begin{packed_item}
			\item Ime
			\item Ime
			\item Korisničko ime
			\item Lozinka
			\item Adresa e-pošte
			\item Broj telefona
	    \end{packed_item}
	    \underline{Registrirani korisnik}, koji osim svih mogućnosti koje ima i neregistrirani korisnik, dodatno ima sljedeće mogućnosti:
		\begin{packed_item}
			\item Postavljanje oglasa o nestalom kućnom ljubimcu 
			\item Sudjelovanje u komunikaciji oko potrage za ljubimcem
			\item Uklanjanje oglasa o nestalom kućnom ljubimcu
			\item Izmjena oglasa o nestalom kućnom ljubimcu
		\end{packed_item}
		Treći tip korisnika su \underline{skloništa za životinje}, koji osiim svih mogućnosti koje ima registrirani korisnik, ima dodatnu mogućnost oglašavanja životinja koje su pronađene i nalaze se u njihovom prostoru. Skloništa za životinje pri registraciji moraju unijeti i naziv skloništa.\\
		
		\textbf{OGLASI}\\
		Za \underline{postavljanje oglasa} potrebno je unijeti sljedeće informacije o ljubimcu:
		\begin{packed_item}
			\item Vrsta
			\item Ime na koje se odaziva
			\item Datum i sat nestanka
			\item Lokacija nestanka
			\item Boja
			\item Starost
			\item Tekstni opis
			\item Slike (najviše 3 slike)
		\end{packed_item}
		Osim podataka o kućnom ljubimcu, oglas sadrži i kontak podatke korisnika koji se povlače iz korisničkih podataka danih pri registraciji (adresa e-pošte, broj telefona).\\
		Oglas mora imati jednu od \underline{kategorija}:
		\begin{packed_item}
			\item Za ljubimcem se traga (aktivan oglas)
			\item Ljubimac je sretno pronađen
			\item Ljubimac nije pronađen, ali se za njim više ne traga aktivno
			\item Ljubimac je pronađen uz nesretne okolnosti
			\item U skloništu (koju može postaviti samo sklonište za životinje)
		\end{packed_item}
		Na oglasu je moguća \underline{izmjena} svih kategorija podataka o ljubimcima, kao i kategorija oglasa. Sve kategorije, osim da se za ljubimcem aktivno traga, oglas čine neaktivnim. Popis neaktivnih oglasa mogu pregledavati samo registrirani korisnici.\\
		\underline{Komunikacija} oko potrage za kućnim ljubimcem odvijat će se porukama koje mogu sadržavati:
		\begin{packed_item}
			\item Tekst
			\item Sliku
			\item Geolokaciju
		\end{packed_item}
		uz kontakt podatke o osobi koja komunicira.\\
		
		\textbf{Ostalo}\\
		Aplikacija je izvedena kao mobilna aplikacija.\\
		Sustav podržava rad više korisnika u stvarnom vremenu.\\
		Aplikacija je namijenjena vlasnicima životinja i svim dobrim dušama koje su volje pomoći oko potrage za nestalim kućnim ljubimcima.
		%\begin{packed_item}
			%\item \textit{potencijalna korist ovog projekta}
			%\item \textit{postojeća slična rješenja (istražiti i ukratko opisati razlike u odnosu na zadani zadatak). Dodajte slike koja predočavaju slična rješenja.}
			%\item \textit{skup korisnika koji bi mogao biti zainteresiran za ostvareno rješenje.}
			%\item \textit{mogućnost prilagodbe rješenja }
			%\item \textit{opseg projektnog zadatka}
			%\item \textit{moguće nadogradnje projektnog zadatka}
		%\end{packed_item}
		
		%\textit{Za pomoć pogledati reference navedene u poglavlju „Popis literature“, a po potrebi konzultirati sadržaj na internetu koji nudi dobre smjernice u tom pogledu.}
		\eject
		
		\section{Primjeri u \LaTeX u}
		
		\textit{Ovo potpoglavlje izbrisati.}\\

		U nastavku se nalaze različiti primjeri kako koristiti osnovne funkcionalnosti \LaTeX a koje su potrebne za izradu dokumentacije. Za dodatnu pomoć obratiti se asistentu na projektu ili potražiti upute na sljedećim web sjedištima:
		\begin{itemize}
			\item Upute za izradu diplomskog rada u \LaTeX u - \url{https://www.fer.unizg.hr/_download/repository/LaTeX-upute.pdf}
			\item \LaTeX\ projekt - \url{https://www.latex-project.org/help/}
			\item StackExchange za Tex - \url{https://tex.stackexchange.com/}\\
		
		\end{itemize} 	


		
		\noindent \underbar{podcrtani tekst}, \textbf{podebljani tekst}, 	\textit{nagnuti tekst}\\
		\noindent \normalsize primjer \large primjer \Large primjer \LARGE {primjer} \huge {primjer} \Huge primjer \normalsize
				
		\begin{packed_item}
			
			\item  primjer
			\item  primjer
			\item  primjer
			\item[] \begin{packed_enum}
				\item primjer
				\item[] \begin{packed_enum}
					\item[1.a] primjer
					\item[b] primjer
				\end{packed_enum}
				\item primjer
			\end{packed_enum}
			
		\end{packed_item}
		
		\noindent primjer url-a: \url{https://www.fer.unizg.hr/predmet/proinz/projekt}
		
		\noindent posebni znakovi: \# \$ \% \& \{ \} \_ 
		$|$ $<$ $>$ 
		\^{} 
		\~{} 
		$\backslash$ 
		
		
		\begin{longtblr}[
			label=none,
			entry=none
			]{
				width = \textwidth,
				colspec={|X[8,l]|X[8, l]|X[16, l]|}, 
				rowhead = 1,
			} %definicija širine tablice, širine stupaca, poravnanje i broja redaka naslova tablice
			\hline \SetCell[c=3]{c}{\textbf{naslov unutar tablice}}	 \\ \hline[3pt]
			\SetCell{LightGreen}IDKorisnik & INT	&  	Lorem ipsum dolor sit amet, consectetur adipiscing elit, sed do eiusmod  	\\ \hline
			korisnickoIme	& VARCHAR &   	\\ \hline 
			email & VARCHAR &   \\ \hline 
			ime & VARCHAR	&  		\\ \hline 
			\SetCell{LightBlue} primjer	& VARCHAR &   	\\ \hline 
		\end{longtblr}
		

		\begin{longtblr}[
				caption = {Naslov s referencom izvan tablice},
				entry = {Short Caption},
			]{
				width = \textwidth, 
				colspec = {|X[8,l]|X[8,l]|X[16,l]|}, 
				rowhead = 1,
			}
			\hline
			\SetCell{LightGreen}IDKorisnik & INT	&  	Lorem ipsum dolor sit amet, consectetur adipiscing elit, sed do eiusmod  	\\ \hline
			korisnickoIme	& VARCHAR &   	\\ \hline 
			email & VARCHAR &   \\ \hline 
			ime & VARCHAR	&  		\\ \hline 
			\SetCell{LightBlue} primjer	& VARCHAR &   	\\ \hline 
		\end{longtblr}
	


		
		
		%unos slike
		\begin{figure}[H]
			\includegraphics[scale=0.4]{slike/aktivnost.PNG} %veličina slike u odnosu na originalnu datoteku i pozicija slike
			\centering
			\caption{Primjer slike s potpisom}
			\label{fig:promjene}
		\end{figure}
		
		\begin{figure}[H]
			\includegraphics[width=\textwidth]{slike/aktivnost.PNG} %veličina u odnosu na širinu linije
			\caption{Primjer slike s potpisom 2}
			\label{fig:promjene2} %label mora biti drugaciji za svaku sliku
		\end{figure}
		
		Referenciranje slike \ref{fig:promjene2} u tekstu.
		
		\eject
		
	